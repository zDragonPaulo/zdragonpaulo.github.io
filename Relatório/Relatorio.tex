\documentclass[a4paper]{article}

\usepackage{amsmath}
\usepackage{graphicx}
\usepackage{subcaption}
\usepackage{setspace}
\usepackage[portuguese]{babel}
\usepackage{siunitx} % Required for alignment
\sisetup{
  round-mode          = places, % Rounds numbers
  round-precision     = 2, % to 2 places
}
\usepackage{enumerate}
\usepackage{enumitem}
\usepackage{amsmath}
\usepackage{karnaugh-map}
\usepackage[section]{placeins}
\usepackage{geometry}
\usepackage{amssymb}
\usepackage{titling}
\usepackage[T1]{fontenc}
\usepackage{float}
\usepackage[hidelinks]{hyperref}
\usepackage{xcolor}
\usepackage{indentfirst}
\usepackage{array}
\usepackage{soul}
\usepackage[style=apa, backend=biber]{biblatex}
\usepackage{csquotes}
\addbibresource{Recursos/referencias.bib}
\newcolumntype{P}[1]{>{\centering\arraybackslash}p{#1}}
%\hypersetup{
%    colorlinks,
%    linkcolor={blue!80!black},
 %   citecolor={blue!50!black},
 %   urlcolor={}
%}

\usepackage{afterpage}
\newcommand\myemptypage{
    \null
    \thispagestyle{empty}
    \addtocounter{page}{-1}
    \newpage
    }


% First title page
% Página de título principal
\newcommand{\firsttitlepage}{
    \begin{titlepage}
        \centering
        \vspace*{1cm}
        
        % Logos superior
        \begin{figure}[h!]
            \centering
            \includegraphics[width=6cm]{Recursos/Logos/LOGO_IPB} % Substitua pelo caminho da imagem
            \vspace{0.5cm}
        \end{figure}

        % Informações da instituição
        \large\textbf{INSTITUTO POLITÉCNICO DE BEJA} \\
        \large\textbf{Escola Superior de Tecnologia e Gestão} \\
        \large\textbf{Licenciatura em Engenharia Informática} \\
        \large\textbf{Dinâmica de Grupos e Comunicação} \\
        
        \vspace{2cm}
        
        % Título do projeto
        {\Huge \textbf{Criação de um Portfólio}} \\
        
        \vspace{1.5cm}
        
        % Autores
        \large Paulo António Tavares Abade - 23919 \\
        
        \vfill
        
        % Logo inferior
        \begin{figure}[h!]
            \centering
            \includegraphics[width=6cm]{Recursos/Logos/IPBejaESTIG.jpg} % Substitua pelo caminho da imagem
        \end{figure}
        
        \vspace{1cm}
        
        % Local e data
        {\large Beja, novembro de 2024}
    \end{titlepage}
}

\newcommand{\secondtitlepage}{
    \begin{titlepage}
        \centering
        \vspace*{1cm}
        
        % Informações da instituição
        \large\textbf{INSTITUTO POLITÉCNICO DE BEJA} \\
        \large\textbf{Escola Superior de Tecnologia e Gestão} \\
        \large\textbf{Licenciatura em Engenharia Informática} \\
        \large\textbf{Dinâmica de Grupos e Comunicação} \\
        
        \vspace{2cm}
        
        % Título do projeto
        {\Huge \textbf{Criação de um Portfólio}} \\
        
        \vspace{1.5cm}
        
        % Autores
        \large Paulo António Tavares Abade - 23919 \\

        \vspace{2cm}

        % Orientador
        \large Orientador:\\ Professora Marta Amaral
        
        \vfill
        
        % Local e data
        {\large Beja, novembro de 2024}
    \end{titlepage}
}

%--------------------------------------------------------------------------------------------------------------------------------------------------------
\begin{document}
\pagenumbering{gobble}


% First title page
\firsttitlepage


% Second title page
\secondtitlepage


\newpage

\renewcommand{\contentsname}{Indíce}
\renewcommand{\refname}{Bibliografia}
\renewcommand{\thefigure}{\thesection.\arabic{figure}}

\newpage
\pagenumbering{roman}
\doublespacing
\tableofcontents
\doublespacing

\newpage
\pagenumbering{arabic}

%--------------------------------------------------------------------------------------------------------------------------------------------------------
\section{Introdução}\label{intro}
Neste relatório será explicado o que é um portfólio e qual a sua importância na vida de um profissional, principalmente nas 
áreas da Tecnologia, já que o nosso trabalho baseia-se na realização de projetos para dar a conhecer as nossas capacidades, 
e os projetos que realizámos no passado ou que estamos a realizar podem ser um fator decisivo para a nossa contratação.

Resumidamente, um portfólio é um documento, geralmente em páginas Web, no caso dos informáticos e designers, que contém
informações sobre o autor, os seus projetos, as suas competências e as suas experiências profissionais. Este documento deve ser 
complementar ao currículo, que é um documento mais formal e que contém informações mais detalhadas sobre o autor.


Para a realização deste trabalho foi criado uma página Web, com o suporte do \textit{Github Pages} (\cite{gitpages})
e o Currículo foi feito com o suporte do \textit{Europass} (\cite{europass}) e o vídeo foi feito com o suporte da aplicação \textit{Câmara} 
da \textit{Microsoft}.

%--------------------------------------------------------------------------------------------------------------------------------------------------------
\newpage
\section{Criação de um Portfólio}\label{criacao}
A criação de um portfólio é uma tarefa que pode ser realizada de várias formas, desde a criação 
de um site utilizando a \textit{WIX} (\cite{WIX}) ou utilizando as tecnologias de desenvolvimento Web, 
como o \textit{HTML}, \textit{CSS} e \textit{JavaScript}. Eu escolhi utilizar o HTML, CSS e JavaScript, 
pois permite mais liberdade a personalizar o site e a adicionar funcionalidades que não são possíveis
com a \textit{WIX} gratuitamente. É mais trabalhoso, porém, o resultado final é mais satisfatório.

Este portfólio é composto por várias secções, como a secção de apresentação onde contêm informações básicas, a secção de projetos recentes, 
interesses, a secção de competências e por fim, a secção de contacto.

Inicialmente tentei criar o portfólio totalmente do zero, sem auxílio de modelos pré-feitos porém, não estava a conseguir o resultado que pretendia,
por isso, decidi utilizar um modelo pré-feito e personalizá-lo ao meu gosto. O colega Raphael Jacuá, sugeriu um Website que disponibiliza vários 
modelos de portfólios gratuitos, o \textit{Free CSS} (\cite{freecss}). Escolhi o modelo \textit{Stacked} (\cite{stacked}) 
e comecei a personalizá-lo.

Ao acessar o site do portfólio, é possível consultar o meu currículo, criado com o \textit{Europass} (\cite{europass}), 
e o vídeo de apresentação, ambos disponíveis na secção de \textit{Currículo}
Este é o link para o portfólio criado: \\
\begin{center} 
    { \large \textbf{\url{https://zdragonpaulo.github.io}}}
\end{center}
Este é o link para o video: \\
\begin{center} 
    { \large \textbf{\url{https://tinyurl.com/videoDGC23919}}}
\end{center}


%--------------------------------------------------------------------------------------------------------------------------------------------------------
\newpage
\section{Conclusões e Perspetivas de Trabalho Futuro}\label{con}
Após a conclusão deste trabalho, posso concluir que a criação de um portfolio é uma tarefa que requer tempo e dedicação,
pois é necessário criar um site que seja apelativo e que contenha todas as informações necessárias para que o empregador possa
avaliar as nossas competências e experiências profissionais. Além disso, é necessário criar um currículo que seja claro e objetivo,
com as informações corretas e relevantes para a vaga a que nos estamos a candidatar. É interessante também adicionar a secção de 
Hobbies e Interesses, pois permite ao empregador conhecer um pouco mais sobre a nossa personalidade e gostos pessoais, percebendo 
assim se somos uma boa opção para a empresa e se nos iremos adaptar facilmente à mesma.

%--------------------------------------------------------------------------------------------------------------------------------------------------------
\newpage
\addcontentsline{toc}{section}{Bibliografia}
\printbibliography


\end{document}