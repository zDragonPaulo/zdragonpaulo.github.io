\documentclass[a4paper]{article}

\usepackage{amsmath}
\usepackage{graphicx}
\usepackage{subcaption}
\usepackage{setspace}
\usepackage[portuguese]{babel}
\usepackage{siunitx} % Required for alignment
\sisetup{
  round-mode          = places, % Rounds numbers
  round-precision     = 2, % to 2 places
}
\usepackage{enumerate}
\usepackage{enumitem}
\usepackage{amsmath}
\usepackage{karnaugh-map}
\usepackage[section]{placeins}
\usepackage{geometry}
\usepackage{amssymb}
\usepackage{titling}
\usepackage[T1]{fontenc}
\usepackage{float}
\usepackage[hidelinks]{hyperref}
\usepackage{xcolor}
\usepackage{indentfirst}
\usepackage{array}
\usepackage{soul}
\usepackage[style=apa, backend=biber]{biblatex}
\addbibresource{Recursos/referencias.bib}
\newcolumntype{P}[1]{>{\centering\arraybackslash}p{#1}}
%\hypersetup{
%    colorlinks,
%    linkcolor={blue!80!black},
 %   citecolor={blue!50!black},
 %   urlcolor={}
%}

\usepackage{afterpage}
\newcommand\myemptypage{
    \null
    \thispagestyle{empty}
    \addtocounter{page}{-1}
    \newpage
    }


% First title page
% Página de título principal
\newcommand{\firsttitlepage}{
    \begin{titlepage}
        \centering
        \vspace*{1cm}
        
        % Logos superior
        \begin{figure}[h!]
            \centering
            \includegraphics[width=6cm]{Recursos/Logos/LOGO_IPB} % Substitua pelo caminho da imagem
            \vspace{0.5cm}
        \end{figure}

        % Informações da instituição
        \large\textbf{INSTITUTO POLITÉCNICO DE BEJA} \\
        \large\textbf{Escola Superior de Tecnologia e Gestão} \\
        \large\textbf{Licenciatura em Engenharia Informática} \\
        \large\textbf{Dinâmica de Grupos e Comunicação} \\
        
        \vspace{2cm}
        
        % Título do projeto
        {\Huge \textbf{Criação de um Portfólio}} \\
        
        \vspace{1.5cm}
        
        % Autores
        \large Martinho José Novo Caeiro - 23917 \\
        
        \vfill
        
        % Logo inferior
        \begin{figure}[h!]
            \centering
            \includegraphics[width=6cm]{Recursos/Logos/IPBejaESTIG.jpg} % Substitua pelo caminho da imagem
        \end{figure}
        
        \vspace{1cm}
        
        % Local e data
        {\large Beja, novembro de 2024}
    \end{titlepage}
}

\newcommand{\secondtitlepage}{
    \begin{titlepage}
        \centering
        \vspace*{1cm}
        
        % Informações da instituição
        \large\textbf{INSTITUTO POLITÉCNICO DE BEJA} \\
        \large\textbf{Escola Superior de Tecnologia e Gestão} \\
        \large\textbf{Licenciatura em Engenharia Informática} \\
        \large\textbf{Dinâmica de Grupos e Comunicação} \\
        
        \vspace{2cm}
        
        % Título do projeto
        {\Huge \textbf{Criação de um Portfólio}} \\
        
        \vspace{1.5cm}
        
        % Autores
        \large Martinho José Novo Caeiro - 23917 \\

        \vspace{2cm}

        % Orientador
        \large Orientador:\\ Professora Marta Amaral
        
        \vfill
        
        % Local e data
        {\large Beja, novembro de 2024}
    \end{titlepage}
}

%--------------------------------------------------------------------------------------------------------------------------------------------------------
\begin{document}
\pagenumbering{gobble}


% First title page
\firsttitlepage


% Second title page
\secondtitlepage


\newpage

\renewcommand{\contentsname}{Indíce}
\renewcommand{\refname}{Webgrafia}
\renewcommand{\thefigure}{\thesection.\arabic{figure}}

\newpage
\pagenumbering{roman}
\doublespacing
\tableofcontents
\doublespacing

\newpage
\pagenumbering{arabic}

%--------------------------------------------------------------------------------------------------------------------------------------------------------
\section{Introdução}\label{intro}
Neste relatório será abordado o que é um portfólio, a sua importância, e a sua criação.

O portfólio é um documento que reúne os trabalhos realizados por um indivíduo, com o objetivo de demonstrar as suas competências e habilidades. 
Este documento é muito importante para o mercado de trabalho, pois permite que o empregador tenha uma ideia clara das capacidades do candidato. 
A criação de um portfólio pode ser feita de várias formas, desde um documento em papel até a um site na internet. 
Neste relatório será abordada a criação de um portfólio em formato digital, que inclui um currículo e um vídeo de apresentação.

%--------------------------------------------------------------------------------------------------------------------------------------------------------
\newpage
\section{Criação de um Portfólio}\label{criacao}
Para a criação de um portfólio foi utilizado o site \textit{Wix} (\cite{wix}), que permite criar um site de forma simples e intuitiva.
Para a criação do CV foi utilizado o site \textit{Europass} (\cite{europass}), que permite criar currículos de forma rápida e com um design apelativo.
Para a criação do vídeo foi utilizada a aplicação \textit{Câmara} da \textit{Microsoft}, que permite criar videos através do computador.

Este é o link para o portfólio criado: \url{https://2391760.wixsite.com/martinhocaeiro} (Na página “Currículo” é possível visualizar o CV e o Vídeo)

%--------------------------------------------------------------------------------------------------------------------------------------------------------
\newpage
\section{Conclusões e Perspetivas de Trabalho Futuro}\label{con}
Após a conclusão do Portfólio + CV + Vídeo conclui que a criação de um documento com as nossas aptidões é extremamente importante 
para o mercado de trabalho, pois permite de uma forma simples e eficaz criar uma boa “primeira impressão” e demonstra que existe 
alguma dedicação face a criação dos mesmos, dai a existência de inúmeras ferramentas que facilitam a criação destes documentos sem 
que seja necessário um nível elevado de experiência. Também sinto que, pelo contrário, o vídeo seja pouco necessário por tratar-se de 
uma repetição, um pouco “forçada”, dos conteúdos presentes nos já referidos documentos, não só, o suporte em vídeo é facilmente 
prejudicado pelo equipamento utilizado.

%--------------------------------------------------------------------------------------------------------------------------------------------------------
\newpage
\addcontentsline{toc}{section}{Webgrafia}
\printbibliography


\end{document}